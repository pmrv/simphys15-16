\RequirePackage[l2tabu,orthodox]{nag} % turn on warnings because of bad style
\documentclass[a4paper,11pt,bibtotoc]{scrartcl}

\usepackage[utf8]{inputenc}
\DeclareUnicodeCharacter{0394}{\Delta}
\usepackage[T1]{fontenc}        % Tries to use Postscript Type 1 Fonts for better rendering
\usepackage{lmodern}            % Provides the Latin Modern Font which offers more glyphs than the default Computer Modern
\usepackage[intlimits]{amsmath} % Provides all mathematical commands

\usepackage{hyperref}           % Provides clickable links in the PDF-document for \ref
\usepackage{grffile}            % Allow you to include images (like graphicx). Usage: \includegraphics{path/to/file}

% Allows to set units
\usepackage[ugly]{units}

% Additional packages
\usepackage{url}                % Lets you typeset urls. Usage: \url{http://...}
\usepackage{breakurl}           % Enables linebreaks for urls
\usepackage{xspace}             % Use \xpsace in macros to automatically insert space based on context. Usage: \newcommand{\es}{ESPResSo\xspace}
\usepackage{xcolor}             % Obviously colors. Usage: \color{red} Red text
\usepackage{booktabs}           % Nice rules for tables. Usage \begin{tabular}\toprule ... \midrule ... \bottomrule

% Source code listings
\usepackage{listings}           % Source Code Listings. Usage: \begin{lstlisting}...\end{lstlisting}
\lstloadlanguages{python}       % Default highlighting set to "python"

\usepackage{epsfig}
\usepackage{cleveref}
\usepackage{subcaption}

\title{Worksheet 5: Thermostats, Monte Carlo}
\author{Evangelos Ribeiro Tzaras \and Marvin Poul}

\begin{document}

\maketitle

\section{Thermostats}

\subsection{Anderson Thermostat}

The Anderson algorithm describes random collisions of the simulated particles
with a heat bath with a frequency $\nu$, as such the higher $\nu$ the more
directly are the system dynamics influence the these thermic collisions rather
than system properties, e.g.\ concentration gradients in the case of the
diffusivity.

\subsection{Nos\' e-Hoover Thermostat}

Choosing $L = 3N + 1$ leads to the ensemble average in the Nos\' e-ensemble
being directly equal the ensemble average in the NVT-ensemble, whereas $L = 3N$
leads to constant time steps.

\subsection{Berendsen Thermostat}

The equipartition theorem gives
\begin{equation*}
    \bar v^2 = 3\frac{k_BT}{m}
\end{equation*}
for the mean velocity at a given temperature. Rescaling all velocities by
$\lambda$ then yields for the kinetic energy
\begin{align*}
    E_\mathrm{kin}' &= \frac{1}{2}m\bar v'^2 = \frac{1}{2}mv^2\lambda^2 \\
    &= \frac{3k_B }{2}\left(T + \frac{\Delta t}{\tau_T}(T' - T)\right) \\
\end{align*}
or, as the difference between before and after the rescaling,
\begin{align*}
    ΔE_\mathrm{kin} &= \frac{3k_B }{2m}\left(T + \frac{\Delta t}{\tau_T}(T' -
    T)\right) - \frac{3}{2}k_BT \\
    &= \frac{3k_B}{2\tau_T}\Delta t\Delta T,
\end{align*}
where $\Delta T = T' - T$. Formulating this as an energy flux $J =
\frac{E}{\Delta t}$ gives the desired expression
\begin{equation*}
    J = \frac{3k_B}{2\tau_T}\Delta T = \alpha \Delta T
\end{equation*}
from which
\begin{equation*}
    \alpha = \frac{3k_B}{2\tau_T}
\end{equation*}
follows directly by comparison.

\section{Simple Sampling --- Integration}

The Runge-function has been plotted in \cref{fig:runge} and the statistical and
actual error in \cref{fig:error}. The code used is in the file \texttt{ex3.py}.

\begin{figure}[htb]
    \centering
    \includegraphics[width=\textwidth]{runge}
    \caption{The Runge-function in the intervall $[-5,5]$}
    \label{fig:runge}
\end{figure}

\begin{figure}[htb]
    \centering
    \includegraphics[width=\textwidth]{error}
    \caption{Errors for different sample sizes, (example run)}
    \label{fig:error}
\end{figure}


\end{document}
