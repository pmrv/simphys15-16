\RequirePackage[l2tabu,orthodox]{nag} % turn on warnings because of bad style
\documentclass[a4paper,11pt,bibtotoc]{scrartcl}

\usepackage[utf8]{inputenc}
\DeclareUnicodeCharacter{0394}{\Delta}
\usepackage[T1]{fontenc}        % Tries to use Postscript Type 1 Fonts for better rendering
\usepackage{lmodern}            % Provides the Latin Modern Font which offers more glyphs than the default Computer Modern
\usepackage[intlimits]{amsmath} % Provides all mathematical commands

\usepackage{hyperref}           % Provides clickable links in the PDF-document for \ref
\usepackage{grffile}            % Allow you to include images (like graphicx). Usage: \includegraphics{path/to/file}

% Allows to set units
\usepackage[ugly]{units}

% Additional packages
\usepackage{url}                % Lets you typeset urls. Usage: \url{http://...}
\usepackage{breakurl}           % Enables linebreaks for urls
\usepackage{xspace}             % Use \xpsace in macros to automatically insert space based on context. Usage: \newcommand{\es}{ESPResSo\xspace}
\usepackage{xcolor}             % Obviously colors. Usage: \color{red} Red text
\usepackage{booktabs}           % Nice rules for tables. Usage \begin{tabular}\toprule ... \midrule ... \bottomrule

% Source code listings
\usepackage{listings}           % Source Code Listings. Usage: \begin{lstlisting}...\end{lstlisting}
\lstloadlanguages{python}       % Default highlighting set to "python"

\usepackage{epsfig}
\usepackage{cleveref}
\usepackage{subcaption}

\title{Worksheet 6: Finite-Size Scaling and the Ising Model}
\author{Evangelos Ribeiro Tzaras \and Marvin Poul}

\begin{document}

\maketitle

\section{Speeding up the Simulation}

First we used the \texttt{GSL} library for random number generation, because we need
this in every MC iteration and it is faster than \texttt{numpy}'s RNG. Next
the inner loops of the MC routine and the initial energy computation were moved
into a Cython module. The former are an obvious optimization target, because
this is where the actual work of the program is performed. The latter is less
obvious because it is only run once per MC run, but profiling revealed it had
never the less an impact. While the error calculation also has potential for
optimization, we decided against moving it into a Cython module, because
profiling showed that even for high numbers of collected samples its runtime is
on average at least an order of magnitude smaller than that of the MC routine
and it is hence not worth the hassle.

\section{Determining Equilibrium Values and Errors}

Results for runs with $L \in \{16, 64\}$ and $T \in [1.0, 5.0, 0.1]$ are
summarized in \cref{fig:energy,fig:magnetization}. Before the phase transition
the magnetization curves are very similar no matter the system size, whereas
the decrease afterwards is sharper and more pronounced the bigger the system
size. For the energy all systems are the same regardless of size.
After it we see no no difference between $L = 16$ and $L = 64$, while both
are slightly higher than $L = 4$. 

\begin{figure}[htb]
    \centering
    \includegraphics[width=.8\textwidth]{magnetization}
    \caption{Energy over Magnetization for different system sizes}
    \label{fig:magnetization}
\end{figure}

\begin{figure}[htb]
    \centering
    \includegraphics[width=.8\textwidth]{energy}
    \caption{Energy over Temperature for different system sizes}
    \label{fig:energy}
\end{figure}

\section{Finite-Size Scaling}

The Binder parameter is plotted in \cref{fig:binder} and all curves cross at $T
= 2.24$.

\begin{figure}[htb]
    \centering
    \includegraphics[width=.8\textwidth]{binder}
    \caption{Binder $U$ over Temperature}
    \label{fig:binder}
\end{figure}

\end{document}
