\RequirePackage[l2tabu,orthodox]{nag} % turn on warnings because of bad style
\documentclass[a4paper,11pt,bibtotoc]{scrartcl}

\usepackage[utf8]{inputenc}
\DeclareUnicodeCharacter{0394}{\Delta}
\usepackage[T1]{fontenc}        % Tries to use Postscript Type 1 Fonts for better rendering
\usepackage{lmodern}            % Provides the Latin Modern Font which offers more glyphs than the default Computer Modern
\usepackage[intlimits]{amsmath} % Provides all mathematical commands

\usepackage{hyperref}           % Provides clickable links in the PDF-document for \ref
\usepackage{grffile}            % Allow you to include images (like graphicx). Usage: \includegraphics{path/to/file}

% Allows to set units
\usepackage[ugly]{units}

% Additional packages
\usepackage{url}                % Lets you typeset urls. Usage: \url{http://...}
\usepackage{breakurl}           % Enables linebreaks for urls
\usepackage{xspace}             % Use \xpsace in macros to automatically insert space based on context. Usage: \newcommand{\es}{ESPResSo\xspace}
\usepackage{xcolor}             % Obviously colors. Usage: \color{red} Red text
\usepackage{booktabs}           % Nice rules for tables. Usage \begin{tabular}\toprule ... \midrule ... \bottomrule

% Source code listings
\usepackage{listings}           % Source Code Listings. Usage: \begin{lstlisting}...\end{lstlisting}
\lstloadlanguages{python}       % Default highlighting set to "python"

\usepackage{epsfig}
\usepackage{cleveref}
\usepackage{subcaption}

\title{Worksheet 4: Error Analysis and Langevin Thermostat}
\author{Evangelos Ribeiro Tzaras \and Marvin Poul}

\begin{document}

\maketitle

\section{Autocorrelation Analysis}

The autocorrelation function for all five series are plotted in \cref{fig:acf}
for all $k < 2000$ and its running sum in \cref{fig:iat}. The
integrated autocorrelation times are put in \cref{tab:auto}. Integrating over
a large interval seems numerically, because in the definition of the
autocorrelation function $\langle A_iA_{i + k}\rangle \simeq \langle A_i\rangle^2$ as the samples become
more and more uncorrelated and we then subtract number of about the same size, which
introduces large errors. This can be seen in the fluctuations in
\cref{fig:acf}. Summing over this just exaggerates this and leads e.g. to
decreasing $\tau$ as seen in series $1$ in \cref{fig:iat} or even negative ones
when larges $k$ are considered.

\begin{figure}[h]
    \centering
    \includegraphics[width=.7\textwidth]{autocorrelation}
    \caption{autocorrelation functions (top) and integrated autocorrelation
    times}
    \label{fig:acf}
    \label{fig:iat}
\end{figure}

\begin{table}
    \centering
    \caption{Results from autocorrelation analysis}
    \label{tab:auto}
    \begin{tabular}{rrrrrr}
        \toprule
    series &  $\langle A \rangle$ &  $\epsilon_A^2$ &  $\tau$ &  $\epsilon_\tau^2$ &  $N_\mathrm{eff}$ \\
    \midrule
    1 & 1.98492 & 0.00048 & 51.45368 & 0.11127 & 1943 \\
    2 & -3.91072 & 0.00324 & 323.25297 & 0.27860 & 309 \\
    3 & 6.00094 & 0.00010 & 9.87739 & 0.04919 & 10124 \\
    4 & -7.99734 & 0.00001 & 0.50460 & 0.01342 & 198177 \\
    5 & 10.01297 & 0.00114 & 112.91853 & 0.16474 & 886 \\
    \bottomrule
    \end{tabular}
\end{table}

\section{Binning analysis}

The binning error of the five sample data series is plotted in
\cref{fig:error-binning}, the blocking $\tau$ in \cref{fig:blocking-tau}.
Eyeballed estimates for the errors and autocorrelation times are given in
\cref{tab:eyeball}, no estimates are given for the second series, because it
does not clearly converge.

\begin{table}
    \centering
    \caption{error and autocorrelation time estimates from graphs}
    \label{tab:eyeball}
    \begin{tabular}{clr}
        \toprule
        Data series & $\epsilon^2$ & $\tau$ \\
        \midrule
        1 & 0.001 & 45 \\
        3 & 0.0002 & 8  \\
        4 & 0.00001 & 0.5 \\
        5 & 0.002 & 100 \\
        \bottomrule
    \end{tabular}
\end{table}

\begin{figure}[htb]
    \centering
    \includegraphics[width=.7\textwidth]{errors-binning}
    \caption{Binning errors over block size}
    \label{fig:error-binning}
\end{figure}

\begin{figure}[htb]
    \centering
    \includegraphics[width=.7\textwidth]{blocking-tau}
    \caption{Blocking $\tau$ over block size}
    \label{fig:blocking-tau}
\end{figure}

\section{Jackknife analysis}

The jackknife error of all series is plotted in \cref{fig:error-jackknife}. The
values from the jackknife analysis match up with the binning analysis pretty
closely, their relative deviation is smaller than $5\,\%$, for this reason the
same estimates are given in \cref{tab:eyeball}.

\begin{figure}[htb]
    \centering
    \includegraphics[width=\textwidth]{errors-jackknife}
    \caption{Jackknife errors over block size}
    \label{fig:error-jackknife}
\end{figure}

\end{document}
